\section{Introduction}

\subsection{Motivation}

 Global software development (GSD) is a reality of today's software development. Companies cross the
 barriers produced by distance, cultural differences, and time zones looking for talented personnel.  
 To succeed in this journey, a new set of challenging problems have to be solved. 
 The challenges include developing software requirements specifications~\cite{Meyer85},
 API design~\cite{NordioMitinMeyerGhezziNittoTamburelli09}, project management, and communication and collaboration
 challenges~\cite{HerbslebMockusETAL00,NordioETAL11}. 


 The effects of GSD have been studied from different angles by researchers in the 
 field~\cite{Carmel1999, NordioETAL11, EstlerNordioFuriaMeyerSchneider12, HolmstromConchuirAgerfalkFitzgerald06, HerbslebMoitra01}. 
 Focusing on communication and the effect of time zones,
 Herbsleb et al.~\cite{HerbslebMockusETAL00} and Espinosa et al.~\cite{EspinosaNingErran07} tried to answer an interesting
 question: how distribution and time zones affect
 distributed software development, specially when the teams are distributed 
 in several countries and continents. 
Nordio et al.~\cite{NordioETAL11}, we studied the effect of time zones and locations on communication 
within distributed teams; they performed the study as part of the DOSE~\cite{NordioMitinMeyer2010,NordioDOSE2011} university course.
Global software development has also been studied from a quality point of view.
 The goal is to identify whether the quality and productivity of distributed projects is comparable to the one produced
 in co-located projects. 
 
 Tools in distributed software development play a key role, however, they are rare. An example is Jazz: a tool for distributed 
 software development that improves 
 communication and collaboration between developers. Jazz does not support project management, so other tools such as 
 Microsoft Project have to be used. Environments for software development, known as Integrated Development Environment (IDE), have been 
 widely developed in the last decade; for example Eclipse and EiffelStudio. While there are tools to support software development 
 and project management, these tools are not integrated in a common environment, and they do not satisfy all the 
 requirements of distributed software development.
 
 \subsection{CloudStudio}

The Integrated Development Environment is the software developer's
central tool. IDEs have undergone considerable advances. While Internet development has
benefitted from IDEs, the IDE has not benefitted from the Internet; it
remains an essentially personal tool, requiring every member of a
project to work on a different copy of the software under development
and periodically to undergo a painful process of reconciliation.



CloudStudio~\cite{NordioEstlerFuriaMeyer11}  brings software
development to the cloud. In recent years ever more human
activities, from banking to text processing, have been ``moved to the
cloud''. CloudStudio does the same for software engineering by
introducing a new paradigm of software development, where all the
products of a software project are shared in a common web-based
repository.


Moving software development to the cloud is not just a matter of
following general trends, but a response to critical software
engineering needs, which current technology does not meet: supporting
today's distributed developments, which often involve teams spread
over many locations, and iterative development practices such as pair
programming and online code reviews; maintaining compatibility between
software elements developed by different team members;
% providing the project manager, at every step, with an accurate view
% of the current project state;
avoiding potentially catastrophic version incompatibility problems;
drastically simplifying configuration management.

CloudStudio brings flexibility to several new facets of software
development, most importantly configuration management (CM): to
replace the traditional and painful update-modify-commit-reconcile
cycle, CloudStudio tracks changes at every location in real time and
displays only the selected users' changes in the integrated editor.
The compiler and other tools are aware of the current user
preferences, and target the version of the code coinciding with the
current view.  CloudStudio also integrates communication tools (a chat
box and Skype), and includes a fully automated verification component,
including both static (proof) and dynamic (testing) tools. This array of tightly 
integrated tools makes CloudStudio an innovative IDE, which
can improve the quality and speed of projects involving distributed
teams, and support highly collaborative development practices.