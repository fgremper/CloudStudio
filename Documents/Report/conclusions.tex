%************************************************************************************
% Authors: Martin Nordio
% Date: March 2011
% Root file: report_root.tex
%************************************************************************************


%----------------------------------------------------------------------------
\chapter{Conclusions}\label{conclusions}
%----------------------------------------------------------------------------






Software engineering is becoming an increasingly distributed activity. Teams are spread out over all of the world and new problems arise, such as the lack of communication and awareness information, which may disrupt progress and jeopardise efficiency and timeliness \cite{ref3}. \\

CloudStudio proposes a new mechanism for making awareness information available and detect conflicts early on. One of the key features of CloudStudio is the opening of its functionality to developers by providing a public and well-documented API. This allows for integration of CloudStudio's awareness information into new services and common IDEs. CloudStudio acts as a separate layer on top existing Git projects and as such can be added at any time, while no specific structure of the Git repository is required. \\

For the implementation of CloudStudio, numerous feature requirements have been set from the start, listed under \ref{designfeatures}. This has been done in order to make sure that the information generated by CloudStudio is useful and accurate. The criteria for success have been specified in a project plan before starting the thesis and have been tightly followed. Furthermore, many new features and a sophisticated web interface have been added. \\

Among the big challenges of the thesis were the distributed nature of the project and making sure all the individual parts work together smoothly, studying the structure of Git and coming up with an awareness system that is useful, realisable and uses the available information from Git repositories. Also, the focus on stability and good error handling required a lot of testing and fixing small bugs. \\

Many extensions to the existing version of CloudStudio are conceivable, some of which are listed in Chapter 5. CloudStudio offers an ideal platform for the "Distributed and Outsourced Software Engineering" (DOSE) course, allowing students to experiment with new sources of awareness information and hopefully improving the workflow of all participants.



