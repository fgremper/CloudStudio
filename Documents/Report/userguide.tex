%************************************************************************************
% Authors: Martin Nordio
% Date: March 2011
% Root file: report_root.tex
%************************************************************************************


%----------------------------------------------------------------------------
\chapter{User Guide}\label{userguide}
%----------------------------------------------------------------------------





\section{Setup and Run Client}

The fastest way to get started with CloudStudio is to use a precompiled client JAR directly from GitHub.

\begin{itemize}

\item Download the latest \texttt{CSClient.jar} directly from: \newline \texttt{http://github.com/fgremper/CloudStudio}

\item Go to \texttt{http://cloudstudio.ethz.ch:7330/} and create a new account by clicking "Sign up" in the top right corner and providing a new username and password.

\item Create a new config.xml file in the same directory as the client JAR you downloaded and paste in the setup configuration.

\item Replace the username and password with your username and password you just created.

\item If you want to work with an existing CloudStudio project, provide its repository alias and the path to your local Git repository, and make sure the repository owner adds you to the repository access list.

\item If you want set up a new CloudStudio project, click "Create repository" in the repository overview and provide an alias, description and possible an URL to a remote repository. Use the repository alias in your configuration file.

\item You can use the client to monitor multiple repositories with the same CloudStudio user account.

\end{itemize}

\section{Using the Web Interface}\label{webinterfaceguide}
